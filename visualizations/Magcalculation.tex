\documentclass{article}
\begin{document}
\author{Katelyn Neese}
\title{Magnitude Calculations for Dark Matter Collisions}
\maketitle


\section{Collisions}       
The following demonstrates a generalized sample calculation for finding the magnitudes of nuclear recoil energies associated with dark matter collisions. There are several assumptions in this particular example:

\begin{itemize}
\item The collision between the nucleus and the dark matter particle is a perfectly elastic, "head-on" collision
\item The nucleus is at rest prior to the collision
\item  The dark matter particle is at rest after the collision 
\end{itemize}

The nucleus, Xenon in this example, is represented by a T (meaning "target") and the dark matter particle is represented by \( \chi \). Furthermore, a prime indicates the veloy after the collision. 

\subsection{Given Variables}

$$v_{T} = 0 \textrm{ km/s}$$
$$v_{\chi } = 220 \textrm{ km/s}$$
$$v^{\prime}_{\chi } = 0 \textrm{ km/s}$$
$$m_{T} = 130 \textrm{ GeV} $$
$$m_{\chi} = 130 \textrm{ GeV} $$

\subsection{Calculations}
The following formula demonstrates the conservation of momentum between the nucleus and dark matter particle:
$$v_{T}m_{T} + v_{\chi}m_{\chi} = v^{\prime}_{T}m_{T} + v^{\prime}_{\chi}m_{\chi} $$
Because the velocity of the nucleus before the collision and the velocity of the dark matter particle after the collision is 0, we now have
$$v_{\chi}m_{\chi} = v^{\prime}_{T}m_{T}$$
The two masses are equal and therefore cancel, leaving us with
$$v_{\chi} = v^{\prime}_{T} = 220 \textrm{ km/s}$$
We now wish to find the energy associated with the recoil of the nucleus, which is given by the kinetic energy equation:
$$E_{k} = \frac{1}{2} mv^{2}$$
Before we can substitute in the values we have for the nucleus, we need to convert from GeV to kgs and from km/s to m/s so that the solution will be in Joules. 
$$1 \textrm{ eV} = 1.783\cdot 10^{-36} \textrm{ kg}$$
$$130 \textrm{ GeV} = 1.3\cdot 10^{11} \textrm{ eV} $$
$$1.3\cdot 10^{11} \textrm{ eV} = 2.317\cdot 10^{-25} \textrm{ kg}$$
$$220\textrm{ km/s} = 2.2\cdot 10^{5}\textrm{ m/s}$$
The above values we can now insert into the kinetic energy equation.
$$E_{k} = \frac{1}{2}(2.317\cdot 10^{-25}\textrm{kg})(2.2\cdot 10^{5}\textrm{m/s})^{2}$$
$$E_{k} = 5.607\cdot 10^{-15}$$
The above answer is in Joules; we need to convert this answer into eV. 
$$1 \textrm{ Joule }= 6.242\cdot 10^{18} \textrm{ eV}$$
$$5.607\cdot 10^{-15}\textrm{ Joules } = 34996 \textrm{ eV }= 34.996\textrm{ keV}$$
The magnitude of the energy of this nucleus' recoil is about 35 keV. 

\subsection{Conclusion}
The above calculation gives an estimate for the order of magnitude of nuclear recoil energies from collisions with dark matter. Other factors, such as the mass of the nucleus, the angle of the collision, or velocity of the dark matter particles relative to the nucleus can increase or decrease this number. This is why dark matter detection experiments are often sensitive to energies ranging from 10 keV to upwards of 200 keV. 

\section{Momentum Transfer}
We can estimate the maximum momentum that can be transferred to the nucleus from a dark matter profile. Like before, we will make several assumptions for this estimation:
\begin{itemize}
\item The collision between the two particles is perfectly elastic
\item The nucleus is at rest prior to the collision
\item The dark matter particle is at rest after the collision
\end{itemize}

In this example, the mass of the nucleus is irrelevant because all of the dark matter particle's momentum will be transferred to it. 

\subsection{Given Variables}
$$v_{T} = 0 \textrm{ km/s}$$
$$v_{\chi } = 220 \textrm{ km/s}$$
$$v^{\prime}_{\chi } = 0 \textrm{ km/s}$$
$$m_{\chi} = 100 \textrm{ GeV} $$
\subsection{Calculations}
In order to determine the magnitude of the momentum transferred, it is easiest to convert to natural units. In this case, the velocity of the dark matter particle can be rounded up to 300 km/s because it is the same order of magnitude as 220 km/s.

$$ v_\chi \simeq 300 \textrm{ km/s}$$
$$ c = 300,000 \textrm{ km/s}$$
$$ v_\chi = .001 c = 10^{-3}c$$

Where c represents the speed of light in km/s . Because the velocity of the dark matter particle is not close to the speed of light, we do not need to take relativity into account. Now, for the momentum calculation:
$$v_{T}m_{T} + v_{\chi}m_{\chi} = v^{\prime}_{T}m_{T} + v^{\prime}_{\chi}m_{\chi} $$
$$v_{\chi}m_{\chi} = v^{\prime}_{T}m_{T}$$
$$(10^{-3}c)\cdot (100 \textrm{ GeV}) = v^{\prime}_{T}m_{T} $$
$$100 \textrm{ MeV} = v^{\prime}_{T}m_{T} $$
Where we assume in the last line that c = 1.
\end{document}